\documentclass{report}

\usepackage{url}
\usepackage{blindtext}
\usepackage{amsmath}
\usepackage{graphicx}
\usepackage{subcaption}
\usepackage{float}
\usepackage{setspace}

\title{Huffman Code File Compression\\159.333 Programming Project}
\date{Summer School\\2017/2018}
\author{Gray Salmond\\ Student ID: 15366621\\
Supervisor: Dr. Andre Barczack}


\begin{document}
\maketitle    
\pagenumbering{gobble}    
\newpage

\abstract
\singlespacing
The report details implementation of Huffman coding as part of an academic computer science project at Massey University. Huffman coding is a method for lossless data compression, the report reviews the literature on the subject to give the reader some background into data compression and Huffman coding in the first chapter of the report. The next chapter explains how the project was implemented and the choices made to create the implementation. In the final chapter results of the implementation are examined and compared to other compression software.
\newpage

\pagenumbering{arabic}


\tableofcontents
\newpage


\listoffigures

\listoftables

\newpage
\doublespacing
\section*{Introduction}


\singlespacing
This report documents an implementation of Huffman coding for a Massey University computer science project. Huffman coding is an example of lossless data compression. Huffman coding makes use of various data structures and information theory to create prefix codes to encode a file. This report reviews relevant literature explains the process application performs to create an encoded file and examines the performance of the implementation.


\doublespacing
\singlespacing
Functional data compression requires an application to be able to encode and decode a data file. Perhaps the most widely known compression format ".zip" files also perform the function on merging multiple files into a single file whilst separating them again upon decoding the file. This report focuses on data compression of a single file.

\chapter{Literature Review}

\section{Data Compression}
\doublespacing


\subsection{What is data compression}


\singlespacing
In the context of data files, data compression is the process of encoding information using a smaller representation than the original file \cite{WikiDataCompression}. For data files the size of the representation can be defined in bits or other derived unit such as bytes.


\doublespacing
\singlespacing
Benifits of data compression occur from reducing the resources required to store and transmit data \cite{WikiDataCompression}. However, in the process of compressing and decompressing data computational resources which may impact benifits gained by compression.


\doublespacing
\singlespacing
The ratio of compression can be defined as:


\begin{equation*}
  C_{r} = \dfrac{d_{1}}{d_{2}}
\end{equation*}


\doublespacing
\singlespacing
The compressed ratio (C$_{r}$) is equal to the compressed file size ($d_{1}$) over the orginal file size ($d_{2}$) \cite{MasseyStudyGuide}.


\doublespacing
\singlespacing
\subsection{Lossless Compression}


\singlespacing


Lossless compression is a method of encoding a file with the goal of representing the file using a smaller amount of bits but not losing any information from the decoding process. The opposite being lossly compression where this is not the case. Huffman coding, of which is the subject of this report is used to implement lossless compression \cite{WikiHuffman}.


\doublespacing
\singlespacing
Lossless compression works because most real world data contains stastical redundancy \cite{WikiDataCompression}. If a file contains symbols that occur more frequently than others we can use smaller representations to represent the frequent symbols and larger representations to represent the less frequent symbols. If a symbol doesn't occur at all in a file then no representation for that symbol is needed.


\doublespacing
\singlespacing
The two most common ways of constructing statistical models for the purposes of data compression are static and adaptive models . Static models build a model after reading all the data and store the model representation in the encoded file. Adapative models update a model as the file is compressed \cite{WikiLossless}. The implemtation of Huffman coding presented in this report is an example of a static model.


\doublespacing
\singlespacing
\subsection{Lossy Compression}


\singlespacing


Lossy compression works by removing less important information from a data file. Audio and Image data are examples of where lossy compression is applied. Human eyes are more sensitive to varitions in luminance than the are to variations in colour \cite{WikiDataCompression}. Because of this fact image compression can remove information from files that humans can not percieve therby improving compression ration for a file by reducing the number of bits needed to represent said file. Similiarly human limitations for hearing various audio signals present opportunties to decrease the need representation of audio files.


\doublespacing
\singlespacing
\subsection{Use cases}


\singlespacing


Various uses for data compression can be defined. General purpose data compression algorithms generally refer to compression algorithms that apply to standard text or binary files. Of these algorithms some may be optimised for a particular type of input file such as text compression . As mentiioned in previous sections specialized audio and image compression algorithms exist. Indeed for any particular domain there may be a case for developing a special purpose compression algorithm, such is the case of HAPZIPPER a compression application for the purpose of compressing genetic data \cite{WikiLossless}.


\doublespacing
\singlespacing


Many of the lossless compression algorithms in use today combine Huffman Coding and other compression algorithms. DEFLATE combines Lempel-Ziv with Huffman coding and is used in zip files, gzip files and PNG images. 

\doublespacing
\singlespacing
\subsection{Entropy}


\singlespacing


Information theory underpins the theoritical background of data compression. Claude Shannon published pioneering  papers on the topic in the 1940s and 1950s \cite{WikiDataCompression}. Shannon came up with the idea information entropy which can be defined by the following equation in regards to binary data \cite{MasseyStudyGuide}:


\begin{equation*}
  H = -\left(\sum\limits_{i=0}^n p_{i} log_{2}(p_{i})\right)
\end{equation*}


\doublespacing
\singlespacing
The negative of the sum of: The probability of the symbol ($p_{i}$) occuring in a file times log base 2 of the probability of the symbol occuring in a file.


\doublespacing
\singlespacing
The implications of information entropy impose a limit on the potential compression ratio. If the likelihood of any symbol is the same entropy will equal the number of bits it takes to represent each symbol in the original file. A lower entropy allows more frequent symbols to be represented in a smaller number of bits.


\doublespacing
\singlespacing
There is no single lossless data compression algorithm that can compress any and all data. In fact it is provably impossible to create such an algorithm \cite{WikiLossless}.


\section{Huffman Coding}

Blah blah \cite{BookVLC}.





\chapter{Implementation}



\section{Background}


\singlespacing
The implementation of Huffman coding describe in this report was implemented in the C++ programming language. There is a wide range of reference material available on Huffman coding and variations on Huffman coding. The application is a simple command line application that runs on OSX or Linux type systems although it could easily be ported to Windows if required. The application described in this report compresses a single file but once again could be extended to allow compressing multiple files into a single file.


\doublespacing
\singlespacing
Huffman coding comprises of two distinct parts. Encoding, taking an uncompressed file and compressing it by building a Huffman tree. And decoding, using a Huffman tree to decompress a file previously compressed using the application.


\doublespacing
\singlespacing
The project is available as open source via GitHub \cite{GitLink} and made use of CMake to build the project and the CLion Integrated Development Environment during development.


\doublespacing
\singlespacing
The implementation has the option of encoding a file and decoding a file at the same time for demonstration purposes. Encoded file filenames are suffixed with .huffCode and decoded files replace the .huffCode suffix with .decoded, the original file is retained for comparison purposes.


\doublespacing
\singlespacing
\section{Encoding}


\singlespacing
Encoding a file using Huffman code requires two passes of the original file. The first pass counts occurrences of a unique symbol within a file and uses this information to build a Huffman tree. A symbol could be any size of bits, for this implementation the initial symbol size is a single byte (8 bits) of which there are 256 possible combinations. The second pass uses the built Huffman tree to build the encoded file by comparing the symbols in the original file to the newly created Huffman codes as defined by the Huffman tree.


\doublespacing
\singlespacing
\subsection{Counting Symbol Frequencies}


\singlespacing
As mentioned previously the first pass involves counting the frequencies of a unique symbol in a file. The implementation represents symbols as a struct, which contains data on which symbol and how many occurrences of the symbol there are. The application to be compressed is examined linearly and the count values are updated as symbols are examined. Much of the literature on Huffman coding uses frequencies instead of counts, however, this implementation uses integer counts instead of percentage frequencies.


\begin{figure}[H]
\caption{Counting occurrences of a basic file}
\includegraphics[scale=0.3]{slide1}
\centering
\end{figure}


\subsection{Building a Huffman tree}


\doublespacing
\singlespacing
To build a Huffman tree a min heap is required, a min heap is a data structure that always has the smallest (in this case the least occurring symbol) element at its root. All the symbols are inserted into the Min-Heap. The project implementation uses its own version of a Min-Heap instead of the standard library for the purpose of gaining experience with data structure implementations.


\doublespacing
\singlespacing
The next step in building a Huffman tree is actually building the tree. To build a Huffman tree we need to create new internal nodes that point to other nodes, either other symbols or other nodes. Upon completion of the Huffman tree, there will be a single root internal node for a binary tree with all symbol nodes as leaves.


\doublespacing
\singlespacing
To build a new internal node the least occurring symbol is removed from the min heap (the root node) but pointed to as a left pointer by the new internal node. Then the next least occurring symbol is then also removed from the min heap (also now the root node) and pointed to by the right pointer of the new internal node. The count of occurrences of the two least occurring symbols are summed and become the count value for the new internal node. Finally, the new internal node is placed back in the min-heap effectively replacing 2 nodes with one. This process is continued until there is the single root node for the Huffman tree.


\begin{figure}[H]
\caption{Creating internal nodes}
\includegraphics[scale=0.3]{slide2}
\centering
\end{figure}


\begin{figure}[H]
\caption{A completed Huffman tree with a single root node}
\includegraphics[scale=0.3]{slide3}
\centering
\end{figure}


\doublespacing
\singlespacing
Once the Huffman tree is built we now have our Huffman codes, this is done by traversing the tree. Going left on the tree is represented as a 0 and going right on the tree is represented as a 1. The project implementation traverses the tree recursively to create a map of symbols to Huffman codes for the final step in the encoding process of actually encoding the file.


\begin{figure}[H]
\caption{Huffman codes ready for encoding file}
\includegraphics[scale=0.3]{slide4}
\centering
\end{figure}


\doublespacing
\singlespacing
There are some challenges in creating an encoded file. The most significant being representing the Huffman tree in the encoded file. For this implementation a system whereby an internal node is represented as a 0 and a leaf is represented by a 1 and followed immediately by its symbol representation was used \cite{GeeksForGeeks}. The implementation can work out from this algorithm when there are no more internal nodes left to traverse.


\doublespacing
\singlespacing
After the tree has been stored in the encoded file the next step is to make a final pass through the file and use the map of symbols and codes created earlier from the Huffman tree to append Huffman codes to the encoded file. A final challenge is presented in that an encoded file might not necessarily be a multiple of 8 bits after the encoding process. To mitigate this a 3-bit header is attached to the front of the encoded file signifying how many bits to ignore (0-7) in the decoding process.


\begin{figure}[H]
\caption{Ignoring extraneous bits}
\includegraphics[scale=0.3]{slide5}
\centering
\end{figure}


\doublespacing
\singlespacing
\section{Decoding}


Decoding a file is relatively simple compared to the process of encoding a file as the most computational effort is performed by the encoding process. To decode an encoded file first the 3-bit header is read, then the Huffman tree representation is read in and built-in memory. Finally, every bit is read in one at a time, if a 0 is encountered we traverse down the left of the tree, if a 1 is encountered we traverse down the right of the tree. If we encounter a leaf node whilst traversing the tree the corresponding symbol is appended to a new decoded file and the traversing process starts from the root node again. This process is continued until the entire file is decoded and the original file is recovered bit for bit.


\begin{figure}[H]
\caption{Begin decoding an encoded Huffman tree}
\includegraphics[scale=0.3]{slide6}
\centering
\end{figure}


\begin{figure}[H]
\caption{Finish decoding an encoded Huffman tree}
\includegraphics[scale=0.3]{slide8}
\centering
\end{figure}


\begin{figure}[H]
\caption{Decoded encoded file}
\includegraphics[scale=0.3]{slide9}
\centering
\end{figure}

\chapter{Results}

\include{chapter3}



\bibliography{myRefs} 
\bibliographystyle{ieeetr}


\end{document}